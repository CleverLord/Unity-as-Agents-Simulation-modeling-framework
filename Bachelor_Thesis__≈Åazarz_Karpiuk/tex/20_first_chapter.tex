\section{Tytuł rozdziału 1}

W rozdziale 1 zwykle dokonuje się przelądu literatury związanej z problemem badawczym pracy. Nie przepisujemy wszystkich książek, jakie wpadną nam w ręce, tylko wybieramy taki materiał, który jest przydatny dla pisanej pracy.

To jest podstawowa sprawa: nie zgubić problemu badawczego i nie pójść w "wątki poboczne". Należy omówić, jak problem jest przedstawiany w literaturze, jakie narzędzia i metody są wykorzystywane do rozwiazywania, w jakich branżach zadanie jest rozwiązywane, jakie wyniki są uzyskiwane itd. Metody i narzędzia można omówić skrótowo lub grupami i rozwinąć opis tych, które znajdą zastosowanie w pracy.


UWAGA! W tym rozdziale zostanie wykorzystane 40\% materiału, który Autor pozna podczas lektury źródeł. Tak po prostu jest!

W tym rozdziale znajdzie się dużo odwołań do literatury i szczęśliwie Overleaf zrobi to automatycznie - cytowania i listę bibliografii załącznikowej. Koniecznie trzeba do references.bib podpiąć listę źródeł zebranych w Mendeley lub w Zotero.

Cytujemy sobie jakieś źródło \cite{Fattahi2020AModels}, drugie \cite{Asefi2019VariableManagement} i jeszcze kolejne \cite{Dolinina2019DevelopmentSettlementsb}. Do bibliografii źródła wpadają w kolejności pojawiania sie w tekście.