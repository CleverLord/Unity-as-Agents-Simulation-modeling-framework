\section{Tytuł rozdziału 2}

W rozdziale 2 zwykle przedstawia się problem badawczy - należy opisać szczegółowo problem, wskazać jego znaczenie dla przedsiębiorstwa/procesu będącego przedmiotem badania, można przedstawić przedsiębiorstwo i jego pozycję na rynku, specyfikę branży (można wspomnieć o znajomości branży przez Autora/Autorkę, jeżeli jest to ważne).

Nalezy przedstawić metodologię badania: plan badania (czyli research protocol), czyli jak zostanie zamodelowany problem (opis + model matematyczny/symulacyjny), sposób generowania zestawów danych, sposób weryfikacji i walidacji modelu.

Tutaj też przezentujemy model matematyczny. Formuły matematyczne w Latex wyglądają daleko lepiej niż w Wordzie :). Poza tym można się do każdej formuły łatwo odwołac, np. do formuły (\ref{formula_01}).

\begin{equation}
\label{formula_01}
    \sum_{i \in I}(a_i*x_{ij}+b_i) \leq c_j, \quad j \in J;
\end{equation}