\section*{Wstęp} \addcontentsline{toc}{section}{Wstęp}

We wstępie formułujemy \textbf{problem badawczy}, czyli jasno przedstawiamy, czym Autor zajmuje się w pracy.

\added[id=JK]{W pracy przedstawiono możliwości silników graficznych, na przykładzie Unity, do realizacji symulacji wieloagentowych.}

Można krótko wspomnieć skąd się wziął pomysł i motywacja do podjęcia problemu -- np. praca zawodowa Autora/Autorki, praktyki zawodowe, zainteresowanie tematem ważnym dla gospodarki, lokalnej społeczności, firmy z którą się współpracuje itp. Jeżeli Autor/Autorka pracuje w danej branży lub z innych powodów zna ją na wylot, należy o tym wspomnieć, bo jest to dodatkowy atut jeżeli chodzi o znajomość problemu. 

\added[id=JK]{Inspiracją na pracę był projekt realizowany w toku nauczania studiów I stopnia z Informatyki i Systemów Inteligentnych i kontynuowany w ramach koła naukowego ``Glider''. Projekt z dziedziny symulacji wieloagentowych dotyczył modelowania zachowania mrówek w uproszczonym środowisku. Symulacje zostały opracowane przez różne grupy z wykorzystaniem innych technologi. Spośród licznych platform Unity, jako przedstawiciel silników graficznych, wyróżniał się znacząco swoimi widocznymi zaletami.}

Na tej podstawie definiujemy hipotezę badawczą oraz stawiamy cel pracy. Cel pracy, np. przeprowadzenie symulacji Monte Carlo dla procesu produkcji czegoś przy różnych wielkościach partii w celu zbadania zależności między wielkością partii a produktywnością.

\added[id=JK]{Hipoteza badawcza pracy brzmi: ``Silniki graficzne stanowią kompetentną platformę do przeprowadzania symulacji wieloagentowych''. \t Celem pracy jest udowodnienie postawionej hipotezy z wykorzystaniem Unity.}

Cel musi być skwantyfikowany (patrz: metodyka wyznaczania celów SMART), czyli trzeba określić miary oceny rozwiązania (KPI), np. zbudowanie, zweryfikowanie i zwalidowanie modelu symulacyjnego, zdefiniowanie scenariuszy badawczych, przeprowadzenie XX uruchmień symulacji w każdym scenariuszu itp. - wszystko w celu uzyskania instancji danych input-output pozwalających przeprowadzić wnioskowania statystyczne.
Hipoteza badawcza (teza), czyli podejrzenie, że istnieje jakiś związek między zmiennymi w problemie i chcemy jego istnienie potwierdzić lub mu zaprzeczyć, np. wielkość partii produkcyjnej ma wpływ na średnią dobową produktywność. 

Metodą badawczą będzie tutaj eksperyment obliczeniowy z wykorzystaniem modelu symulacyjnego, czyli przeprowadzenie symulacji MC w programie X albo rozwiazanie problemu z wykorzystaniem modelu MIP.

Dalej idzie standardowa formułka, że układ pracy jest następujący. W rozdziale 1 przedstawiono ....... W rozdziale 2 ..... W rozdziale 3 .....

\added[id=JK]{W rozdziale pierwszym przedstawiono ..., }